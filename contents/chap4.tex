\section{收敛性证明}
\begin{lemma}
   对任意非零向量 $\mathbf{u},\mathbf{u}^p\in\mathbb{R}^1$,有以下不等式:
   \begin{equation}
   \|\mathbf{u}\|_{1}-\frac{\|\mathbf{u}\|^{2}_{1}}{2\|\mathbf{u}^p\|_{1}}\le
   \|\mathbf{u}^p\|_{1}-\frac{\|\mathbf{u}^p\|^{2}_{1}}{2\|\mathbf{u}^p\|_{1}}
   \label{yw1}
   \end{equation}
\end{lemma}

证明:
\begin{equation}
\begin{aligned}
&(\sqrt{\mathbf{v}}-\sqrt{\mathbf{v^p}})^2\ge0
\Rightarrow\mathbf{v}-2\sqrt{\mathbf{vv}^p}+\mathbf{v}^p\ge0\\
&\Rightarrow\sqrt{\mathbf{v}}-\frac{\mathbf{v}}{2\sqrt{\mathbf{v}^p}}\le
\frac{\sqrt{\mathbf{v}^p}}{2}
\Rightarrow\sqrt{\mathbf{v}}-\frac{\mathbf{v}}{2\sqrt{\mathbf{v}^p}}\le
\sqrt{\mathbf{v}^p}-\frac{\mathbf{v}^p}{2\sqrt{\mathbf{v}^p}}
\end{aligned}
\label{yw2}
\end{equation}
将式 \eqref{yw2} 中的 $\mathbf{v}$ 和 $\mathbf{v}^p$ 替换为 $\|\mathbf{u}\|^{2}_{1}$ 和 $\|\mathbf{u}^p\|^{2}_{1}$,即得到式 \eqref{yw1}。

\begin{theorem}
   算法1在每步迭代中都使问题 的目标值单调递减。
\end{theorem}

证明:首先,用以下等式重写\ 中的问题:
\begin{equation}
\mathbf{z}^{(p+1)}_{1}=\mathop{\arg\min}_{\mathbf{z}_{1}}
\frac{1}{2}\mathbf{z}^{T}_{1}\mathbf{H}^{T}\mathbf{D}^{T}_{1}\mathbf{H}\mathbf{z}_{1}+
c_{1}\mathbf{e}^{T}_{2}\mathop{\max}(0,\mathbf{e}_{2}+\mathbf{Gz}_{1})
\label{yw3}
\end{equation}
即:
\begin{equation}
\mathbf{z}^{(p+1)}_{1}=\mathop{\arg\min}_{\mathbf{z}_{1}}
\frac{1}{2}(\mathbf{Hz}_{1})^{T}\mathbf{D}^{p}_{1}\mathbf{Hz}_{1}+
c_{1}\mathbf{e}^{T}_{2}\mathop{\max}(0,\mathbf{e}_{2}+\mathbf{Gz}_{1})
\label{yw4}
\end{equation}
因此,在第 $(p+1)$ 步迭代中,有
\begin{equation}
\begin{aligned}
&\frac{1}{2}(\mathbf{Hz}^{(p+1)}_{1})^{T}\mathbf{D}^{p}_{1}(\mathbf{Hz}^{(p+1)}_{1})+
c_{1}\mathbf{e}^{T}_{2}\mathop{\max}(0,\mathbf{e}_{2}+\mathbf{Gz}^{(p+1)}_{1})\\
&\le\frac{1}{2}(\mathbf{Hz}^{p}_{1})^{T}\mathbf{D}^{p}_{1}(\mathbf{Hz}^{p}_{1})+
c_{1}\mathbf{e}^{T}_{2}\mathop{\max}(0,\mathbf{e}_{2}+\mathbf{Gz}^{p}_{1})
\end{aligned}
\label{yw5}
\end{equation}
将式 \eqref{yw1} 中的 $\mathbf{u}$ 和 $\mathbf{u}^p$ 替换为 $\mathbf{Hz}^{(p+1)}_{1}$ 和 $\mathbf{Hz}^{p}_{1}$,可以得到:
\begin{equation}
\|\mathbf{Hz}^{(p+1)}_{1}\|_{1}-\frac{\|\mathbf{Hz}^{(p+1)}_{1}\|^{2}_{1}}{2\|\mathbf{Hz}^{p}_{1}\|_{1}}\le
\|\mathbf{Hz}^{p}_{1}\|_{1}-\frac{\|\mathbf{Hz}^{p}_{1}\|^{2}_{1}}{2\|\mathbf{Hz}^{p}_{1}\|_{1}}
\label{yw6}
\end{equation}
因此可得如下不等式:
\begin{equation}
\sum^{m_1}_{i=1}(\left|\mathbf{h}^{T}_{i}\mathbf{z}^{(p+1)}_1\right|-
\frac{(\mathbf{h}^{T}_{i}\mathbf{z}^{(p+1)}_1)^2}
{2\left|\mathbf{h}^{T}_{i}\mathbf{z}^{p}_{1}\right|})\le
\sum^{m_1}_{i=1}(\left|\mathbf{h}^{T}_{i}\mathbf{z}^{p}_1\right|-
\frac{(\mathbf{h}^{T}_{i}\mathbf{z}^{p}_1)^2}
{2\left|\mathbf{h}^{T}_{i}\mathbf{z}^{p}_{1}\right|})
\label{yw7}
\end{equation}
该式可被简化为:
\begin{equation}
\begin{aligned}
&\|\mathbf{Hz}^{(p+1)}_{1}\|_{1}-\frac{1}{2}(\mathbf{Hz}^{(p+1)}_{1})^{T}\mathbf{D}^{p}_{1}(\mathbf{Hz}^{(p+1)}_{1})\\
&\le\|\mathbf{Hz}^{p}_{1}\|_{1}-\frac{1}{2}(\mathbf{Hz}^{p}_{1})^{T}\mathbf{D}^{p}_{1}(\mathbf{Hz}^{p}_{1})     \end{aligned}
\label{yw8}
\end{equation}
综合式 \eqref{yw5} 和 \eqref{yw8},可得:
\begin{equation}
\begin{aligned}
&\|\mathbf{Hz}^{(p+1)}_{1}\|_{1}+c_{1}\mathbf{e}^{T}_{2}\mathop{\max}(0,\mathbf{e}_{2}+\mathbf{Gz}^{(p+1)}_{1})\\
&\le\|\mathbf{Hz}^{p}_{1}\|_{1}+c_{1}\mathbf{e}^{T}_{2}\mathop{\max}(0,\mathbf{e}_{2}+\mathbf{Gz}^{p}_{1})
\end{aligned}
\label{yw9}
\end{equation}
因为\ 中的问题恒小于零,因此算法1收敛,\eqref{yw9} 中的不等式成立。这表示\ 中的目标值随迭代递减,直到算法收敛。
\begin{theorem}
算法1收敛至问题\ 的一个局部最优解。
\end{theorem}
证明:问题\ 的拉格朗日函数如下:
\begin{equation}
L_{2}(\mathbf{z}_{1},\mathbf{q}_{1})=\frac{1}{2}\|\mathbf{Hz}_{1}\|_{1}+
c_{1}\mathbf{e}^{T}_{2}\mathbf{q}_{1}-\mathbf{\alpha}^{T}(-\mathbf{Gz}_{1}+\mathbf{q}_{1}-\mathbf{e}_{2})
-\mathbf{\beta^{T}q}_{1}
\label{yw10}
\end{equation}
其中,$\mathbf{\alpha}$ 和 $\mathbf{\beta}$ 是拉格朗日乘子向量。通过对其求导并取零,可以得到问题\ 的 KKT 条件:
\begin{equation}
\mathbf{H}^{T}\mathbf{D}_{1}\mathbf{Hz}_{1}+\mathbf{G\alpha}=0,
c_{1}\mathbf{e}_{2}-\mathbf{\alpha}-\mathbf{\beta}=0
\label{yw11}
\end{equation}
在算法1的每步迭代中,寻找问题\ 中的最优 $\mathbf{z}^{(p+1)}_{1}$ 。因此,算法1的收敛解满足问题的 KKT 条件。接下来,定义算法1中问题\ 的拉格朗日函数如下:
\begin{equation}
L_{3}(\mathbf{z}_{1},\mathbf{q}_{1})=\frac{1}{2}\mathbf{H}^{T}\mathbf{D}_{1}\mathbf{Hz}_{1}+
c_{1}\mathbf{e}^{T}_{2}\mathbf{q}_{1}-\mathbf{\alpha}^{T}(-\mathbf{Gz}_{1}+\mathbf{q}_{1}-\mathbf{e}_{2})
-\mathbf{\beta^{T}q}_{1}
\label{yw12}
\end{equation}
同样对其求导并取零,得到:
\begin{equation}
\mathbf{H}^{T}\mathbf{D}_{1}\mathbf{Hz}_{1}+\mathbf{G\alpha}=0,
c_{1}\mathbf{e}_{2}-\mathbf{\alpha}-\mathbf{\beta}=0
\label{yw13}
\end{equation}
根据算法1中 $\mathbf{D}_{1}$ 的定义,等式 \eqref{yw11} 和 \eqref{yw13} 在算法1收敛时成立。这说明算法1的收敛解 $\mathbf{z}^{(p+1)}_{1}$ 满足问题\ 的KKT条件,是问题\ 的一个局部最优解。
