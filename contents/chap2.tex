\section{L1-TWSVM}
TWSVM 有良好的分类性能,已成为数据分类研究的热点。但 TWSVM 使用对离群值较为敏感的 L2 范数来度量距离,导致异常观测点可能会对其结果有较大影响。由于 L1 范数是 L2 范数距离的鲁棒替代,\parencite{yan2018efficient} 提出基于 L1 范数的鲁棒分类器。优化问题如下:
\begin{align}
\begin{split}
\label{ts3}
	\min\limits_{\mathbf{w}_1,b_1} \;& \frac{1}{2}||\mathbf{Aw}_1+\mathbf{e}_1b_1||_1+c_1\mathbf{e}_2^T\mathbf{q}_1 \\
	s.t.\;& -(\mathbf{Bw}_1+\mathbf{e}_2b_1)+\mathbf{q}_1 \geq \mathbf{e}_2,\mathbf{q}_1\geq 0
\end{split}
\\
\begin{split}
\label{ts4}
	\min\limits_{\mathbf{w}_2,b_2} \;& \frac{1}{2}||\mathbf{Bw}_2+\mathbf{e}_2b_2||_1+c_2\mathbf{e}_1^T\mathbf{q}_2 \\
	s.t. \; &(\mathbf{Aw}_2+\mathbf{e}_1b_2)+\mathbf{q}_2\geq \mathbf{e}_1, \mathbf{q}_2\geq 0
\end{split}	
\end{align}
其中 $||\cdot||_1$ 表示 L1 范数。在最小化目标函数时,每个平面要尽可能靠近两个分类中的一类,并尽可能远离另一类。由于公式 (\ref{ts3})、(\ref{ts4}) 中不等式为非凸约束,具有局部最优解,可以求解得到两个不平行的超平面:
\begin{align}
	\mathbf{x}^T\mathbf{w}_1+b_1=0, \mathbf{x}^T\mathbf{w}_2+b_2=0
\end{align}
则原问题可优化为:
\begin{align}
\begin{split}
\label{ts5}
	\min\limits_{\mathbf{w}_1,b_1} \;& \frac{1}{2}(\sum_{i=1}^{m_1}\frac{(\mathbf{a}_i^T\mathbf{w}_1+e_1^ib_1)^2}{d_i})+c_1\mathbf{e}_2^T\mathbf{q}_1  \\
	s.t.\;& -(\mathbf{Bw}_1+\mathbf{e}_2b_1)+\mathbf{q}_1 \geq \mathbf{e}_2,\mathbf{q}_1\geq 0
\end{split}
\\
\begin{split}
	\min\limits_{\mathbf{w}_2,b_2} \;& \frac{1}{2}(\sum_{j=1}^{m_2}\frac{(\mathbf{b}_j^T\mathbf{w}_2+e_2^jb_2)^2}{d_j})+c_2\mathbf{e}_1^T\mathbf{q}_2  \\
	s.t. \; &(\mathbf{Aw}_2+\mathbf{e}_1b_2)+\mathbf{q}_2\geq \mathbf{e}_1, \mathbf{q}_2\geq 0
\end{split}
\end{align}
其中 $d_i=|\mathbf{a}_i^T\mathbf{w}_1+e_1^ib_1|\ne 0$,$d_j=|\mathbf{b}_j^T\mathbf{w}_2+e_2^jb_2|\ne 0$,$e_1^i,e_2^j$ 分别表示 $e_1$ 的第 $i$ 个元素和 $e_2$ 的第 $j$ 个元素。由于上述两个式子都包含绝对值运算,难以直接求解,本文提出了一种迭代凸优化策略,基本思想为迭代更新增广向量 $z_1$ 直到连续两次迭代式 (\ref{ts5}) 的目标值小于一个固定值(如0.001),则 $z_1$ 为局部最优解。记 $z_1^p$ 为第 $p$ 次迭代结果,则第 $p+1$ 次迭代结果 $z_1^{(p+1)}$ 可等价为下述问题的解:
\begin{align}
\begin{split}
\label{ts6}
	\min\limits_{\mathbf{z}_1} \;& \frac{1}{2}(\sum_{i=1}^{m_1}\frac{(\mathbf{h}_i^T\mathbf{z}_1)^2}{d_{1i}})+c_1\mathbf{e}_2^T\mathbf{q}_1 \ \\
	s.t.\;& -(\mathbf{Gz}_1+\mathbf{q}_1) \geq \mathbf{e}_2,\mathbf{q}_1\geq 0
\end{split}
\\
\begin{split}
\label{ts7}
	\min\limits_{\mathbf{z}_2} \;& \frac{1}{2}(\sum_{j=1}^{m_2}\frac{(\mathbf{g}_j^T\mathbf{z}_2)^2}{d_{2j}})+c_2\mathbf{e}_1^T\mathbf{q}_2  \\
	s.t. \;& (\mathbf{Hz}_2+\mathbf{q}_2)\geq \mathbf{e}_1, \mathbf{q}_2\geq 0
\end{split}	
\end{align}
其中 $d_{1i}=|\mathbf{h}_i^T\mathbf{z}_1^p|$,$d_{2j}=|\mathbf{g}_j^T\mathbf{z}_2^p|$,$\mathbf{g}_j^T=(\mathbf{b}_j^Te_2^j)$,则公式 (\ref{ts6})、(\ref{ts7}) 可改写为
\begin{align}
\begin{split}
\label{ts8}
	\min\limits_{\mathbf{z}_1} \;& \frac{1}{2}\mathbf{z}_1^T\mathbf{H}^T\mathbf{D}_1\mathbf{Hz}_1+c_1\mathbf{e}_2^T\mathbf{q}_1 \\
	s.t.\;& -(\mathbf{Gz}_1+\mathbf{q}_1) \geq \mathbf{e}_2,\mathbf{q}_1\geq 0
\end{split}
\\
\begin{split}
\label{ts9}
	\min\limits_{\mathbf{z}_2} \;& \frac{1}{2}\mathbf{z}_2^T\mathbf{H}^T\mathbf{D}_2\mathbf{Gz}_2+c_2\mathbf{e}_1^T\mathbf{q}_2 \\
	s.t. \;& (\mathbf{Hz}_2+\mathbf{q}_2)\geq \mathbf{e}_1, \mathbf{q}_2\geq 0
\end{split}
\end{align}
其中 $\mathbf{D}_1=diag(1/d_{11},1/d_{12},\ldots,1/d_{1m_1}),\,\mathbf{D}_2=diag(1/d_{21},1/d_{22},\ldots,1/d_{2m_2})$ 为对角矩阵。
则问题 (\ref{ts8})、(\ref{ts9}) 等价于:
\begin{align}
\begin{split}
	\min\limits_{\mathbf{z}_1} \;& \frac{1}{2}||\mathbf{Hz}_1||_1+c_1\mathbf{e}_2^T\mathbf{q}_1 \\
	s.t.\;& -(\mathbf{Gz}_1+\mathbf{q}_1) \geq \mathbf{e}_2,\mathbf{q}_1\geq 0
\end{split}
\\
\begin{split}
	\min\limits_{\mathbf{z}_2} \;& \frac{1}{2}||\mathbf{Gz}_2||_1+c_2\mathbf{e}_1^T\mathbf{q}_2 \\
	s.t. \;& (\mathbf{Hz}_2+\mathbf{q}_2)\geq \mathbf{e}_1, \mathbf{q}_2\geq 0
\end{split}
\end{align}
公式 (\ref{ts3}) 是不等式约束(非凸)的凸优化问题,因此它存在解析解。其拉格朗日函数为:
\begin{align}
\begin{split}
	\pmb{L}_1(\mathbf{w}_1,b_1,\mathbf{q}_1,\pmb{\alpha},\pmb{\beta}) &=\frac{1}{2}(\mathbf{Aw}_1+\mathbf{e}_1b_1)^T\mathbf{D}_1(\mathbf{Aw}_1+\mathbf{e}_1b_1)\\
	&+c_1\mathbf{e}_2^T\mathbf{q}_1-\pmb{\alpha}^T(-(\mathbf{Bw}_1+\mathbf{e}_2b_1)+\mathbf{q}_1-\mathbf{e}_2)-\pmb{\beta}^T\mathbf{q}_1
\end{split}
\end{align}
其中 $\pmb{\alpha}=(\alpha_1,\alpha_2,\alpha_3,\ldots,\alpha_{m_2})^T, \pmb{\beta}=(\beta_1,\beta_2,\beta_3,\ldots,\beta_{m_1})^T$ 为拉格朗日乘子,$\pmb{\alpha}\geq 0,\pmb{\beta}\geq 0$,令 $\pmb{L}_1$对$\mathbf{w}_1,b_1,\mathbf{q}_1$ 的偏导分别为0,可得Karush-Kuhn-Tucker (KKT)条件为:
\begin{gather}
    \label{ts019}
	\frac{\partial{L}}{\partial{\mathbf{w}_1}}=\mathbf{A}^T\mathbf{D}_1(\mathbf{Aw}_1+\mathbf{e}_1b_1)+\mathbf{B}^T\pmb{\alpha}=0\\
    \label{ts020}
	\frac{\partial{L}}{\partial{b_1}}=\mathbf{e}_1^T\mathbf{D}_1(\mathbf{Aw}_1+\mathbf{e}_1b_1)+\mathbf{e}_2\pmb{\alpha}=0\\
	\label{ts10}
	\frac{\partial{L}}{\partial{\mathbf{q}_1}}=c_1\mathbf{e}_2-\pmb{\alpha}-\pmb{\beta}=0\\
	-(\mathbf{Bw}_1+\mathbf{e}_2b_1)+\mathbf{q}_1 \geq \mathbf{e}_2, \mathbf{q}_1\geq 0\\
	\pmb{\alpha}^T(-(\mathbf{Bw}_1+\mathbf{e}_2b_1)+\mathbf{q}_1-\mathbf{e}_2)=0,\,\pmb{\beta}^T\mathbf{q}_1=0
\end{gather}
从式 (\ref{ts10}) 可推出 $0\le \pmb{\alpha} \le c_1\mathbf{e}_2$,结合公式(\ref{ts019})、(\ref{ts020})可得:
\begin{align}
	\mathbf{A}^T\mathbf{e}_1^T\mathbf{D}_1(\mathbf{Ae}_1)(\mathbf{w}_1b_1)^T+(\mathbf{B}^T\mathbf{e}_2^T)\pmb{\alpha}=0
\end{align}
结合之前定义的矩阵 $(\mathbf{H,G})$ 及增广向量 $(\mathbf{z}_1,\mathbf{z}_2)$,可以得到
\begin{align}
	\mathbf{H}^T\mathbf{D}_1^p\mathbf{Hz}_1^{(p+1)}+\mathbf{G}^T\pmb{\alpha}=0
\end{align}
即
\begin{align}
\label{ts11}
	\mathbf{z}_1^{(p+1)}=-(\mathbf{H}^T\mathbf{D}_1^p\mathbf{H})^{-1}\mathbf{G}^T\pmb{\alpha}
\end{align}
由于 $(\mathbf{H}^T\mathbf{D}_1^p\mathbf{H})^{-1}$ 为半正定矩阵,因此可能得到不稳定或不准确的解,在实际应用中,本文使用正则化方法解决这个问题。$(\mathbf{H}^T\mathbf{D}_1^p\mathbf{H}+\varepsilon \mathbf{I})$ 为正定矩阵 (其中 $\varepsilon$ 为一个小扰动),不受奇点的影响。则逆矩阵 $(\mathbf{H}^T\mathbf{D}_1^p\mathbf{H})^{-1}$ 可由 $(\mathbf{H}^T\mathbf{D}_1^p\mathbf{H}+\varepsilon \mathbf{I})$ 代替,因此 $\mathbf{z}_1^{(p+1)}$ 可推导为:
\begin{align}
	\mathbf{z}_1^{(p+1)}=-(\mathbf{H}^T\mathbf{D}_1^p\mathbf{H}+\varepsilon \mathbf{I})^{-1}\mathbf{G}^T\pmb{\alpha}
\end{align}
同样的
\begin{align}
	\mathbf{z}_2^{(p+1)}=-(\mathbf{G}^T\mathbf{D}_2^p\mathbf{G}+\varepsilon \mathbf{I})^{-1}\mathbf{H}^T\pmb{\beta}
\end{align}
将增广向量 $\mathbf{z}_1^{(p+1)},\mathbf{z}_2^{(p+1)}$ 分别代入拉格朗日函数中。在KKT条件下,原问题 (\ref{ts3})、(\ref{ts4}) 转变为 Wolfe 对偶问题:
\begin{align}
\begin{split}
	\max \limits_{\pmb{\alpha}} \;& \mathbf{e}_2^T\pmb{\alpha}-\frac{1}{2}\pmb{\alpha}^T\mathbf{G}(\mathbf{H}^T\mathbf{D}_1\mathbf{H})^{-1}\mathbf{G}^T\pmb{\alpha} \\
	s.t.\;& 0\le \pmb{\alpha} \le c_1\mathbf{e}_2
\end{split}
\\
\begin{split}		
	\max \limits_{\pmb{\beta}} \;& \mathbf{e}_1^T\pmb{\beta}-\frac{1}{2}\pmb{\beta}^T\mathbf{H}(\mathbf{G}^T\mathbf{D}_2\mathbf{G})^{-1}\mathbf{H}^T\pmb{\beta} \\
	s.t.\;& 0\le \pmb{\beta} \le c_2\mathbf{e}_1
\end{split}
\end{align}
通过求解对偶问题可得拉格朗日乘子 $\pmb{\alpha} \in \pmb{R}^{m_2\times 1},\pmb{\beta} \in \pmb{R}^{m_1\times 1}$ 以及权向量 $\mathbf{w}_1,\mathbf{w}_2$、偏差 $b_1,b_2$,即获得两个不平行的超平面。
对于新加入的点 $\mathbf{x}\in \mathbf{R}^n$,根据决策方程 $f(\mathbf{x})$ (选择最近的超平面) 将其分配到对应类别中
\begin{align}
	f(\mathbf{x})=\mathop{\arg\min}\limits_{i=1,2} \;(|\mathbf{x}^T\mathbf{w}_i+b_i|/|||\mathbf{w}_i|)
\end{align}
其中 $|\cdot|$ 表示取绝对值。
式 (\ref{ts3}) 中的目标函数为非凸约束的凸问题,因此 $\mathbf{z}_1^{(p+1)}$ 为此问题的局部最优解。而在式 (\ref{ts11}) 中,$\mathbf{D}_1^p$ 依赖于 $\mathbf{z}_1^{(p+1)}$,因此它是一个未知变量,可看作 (\ref{ts3}) 中目标的隐变量,可以同相同的迭代算法交替优化求解。我们根据前一次迭代结果 $\mathbf{z}_1^{(p+1)}$ 来更新 $\mathbf{D}_1^p$,又通过 $\mathbf{D}_1^p$ 来改变 $\mathbf{z}_1^{(p+1)}$,增加p直到连续两次迭代结果小于一个固定值。此外,适当的初始化可有效加快算法收敛速度。本文通过求解公式 (\ref{ts1})、(\ref{ts2})得到初始解,仿真结果较优。算法1总结了L1-TWSVM的迭代过程。

\begin{algorithm}[H]
 \textbf{Input}: $A \in \mathbb{R}^{m_1 \times n}$ and $B \in \mathbb{R}^{m_2 \times n}$\;
 Construct the matrices $H = (A\; e_1)$ and $G = (B\; e_2)$\;
 Set $p = 0$. Initialize $z^p$, a standard solution of TWSVM\;
 \While{not converge}{
  Compute $D_1^p$\;
  Compute $z_1^{(p+1)}$ by solving
  \begin{equation}
  \label{algo1}
  z_1^{(p+1)} = \mathop{\arg\min}_{z_1} \frac{1}{2}z_1^TH^TD_1^pHz_1 + c_1e_2^Tq_1,\;s.t.\,-Gz_1+q_1\geq e_2,\,q_1\geq 0
  \end{equation}
  p=p+1;
 }
 \textbf{Output}: The learned solution of $z_1$.
 \caption{L1-TWSVM}
\end{algorithm}