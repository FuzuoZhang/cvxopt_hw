\documentclass{article}
\usepackage[UTF8]{ctex}
\usepackage{amsmath}
\usepackage{amsthm}
\usepackage{amssymb}
\usepackage{float}
\usepackage{graphicx}
\usepackage{color}
%\include{macros}
%\usepackage{floatflt}
%\usepackage{graphics}
%\usepackage{epsfig}
\usepackage[colorlinks,linkcolor=red,anchorcolor=blue,citecolor=green]{hyperref}
\usepackage{epstopdf}
\usepackage{subfigure}
\usepackage{caption}

\theoremstyle{definition}
\newtheorem{theorem}{Theorem}[section]
\newtheorem{lemma}[theorem]{Lemma}
\newtheorem{proposition}[theorem]{Proposition}
\newtheorem{corollary}[theorem]{Corollary}

\theoremstyle{definition}
\newtheorem*{definition}{Definition}
\newtheorem*{example}{Example}

\theoremstyle{remark}
\newtheorem*{remark}{Remark}
\newtheorem*{note}{Note}
\newtheorem*{exercise}{Exercise}

\setlength{\oddsidemargin}{-0.25 in}
\setlength{\evensidemargin}{-0.25 in} \setlength{\topmargin}{-0.25
in} \setlength{\textwidth}{7 in} \setlength{\textheight}{8.5 in}
\setlength{\headsep}{0.25 in} \setlength{\parindent}{0 in}
\setlength{\parskip}{0.1 in}

\newcommand{\homework}[4]{
\pagestyle{myheadings} \thispagestyle{plain}
\newpage
\setcounter{page}{1} \setcounter{section}{0} \noindent
\begin{center}
\framebox{ \vbox{\vspace{2mm} \hbox to 6.28in { {\bf
THU-70250403,~Convex~Optimization~(Fall 2018) \hfill Homework: #4}}
\vspace{6mm} \hbox to 6.28in { {\Large \hfill #1 \hfill} }
\vspace{6mm} \hbox to 6.28in { {\it Lecturer: #2 \hfill} }
\vspace{2mm} \hbox to 6.28in { {\it Student: #3 \hfill} }
\vspace{2mm} } }
\end{center}
\markboth{#1}{#1} \vspace*{4mm} }


\begin{document}

\homework{L1-TWSVM}{王书宁,李力 \hspace{5mm} {\tt swang@tsinghua.edu.cn \tt
li-li@tsinghua.edu.cn }}{张芙作\hspace{5mm} {\tt zhangfz15@mails.tsinghua.edu.cn} }{2}

\section{TWSVM}

TWSVM(孪生支持向量机)是Jayadeva等人于2007年提出的一种改进的双分界面支持向量机,用于解决二分类问题。与传统的支持向量机(SVM)不同,TWSVM为每一类的数据点单独建立一个分类面,其优化策略为,使同一类的数据点尽可能集中的围绕在该类分类面的周围,并且远离另一类数据的分类面。所以TWSVM需要解决两个二次规划问题,得到两个不平行的分类面,但是同一类的数据要作为另一个二次规划问题的约束条件,反之亦然。

TWSVM需要求解以下两个二次优化问题:

\begin{align}
&\min\limits_{\mathbf{w}_1,b_1} \; \frac{1}{2}||\mathbf{Aw}_1+\mathbf{e}_1b_1||_2^2+c_1\mathbf{e}_2^T\mathbf{q}_1 \nonumber \\
&s.t.\; -(\mathbf{Bw}_1+\mathbf{e}_2b_1)+\mathbf{q}_1 \geq \mathbf{e}_2,\mathbf{q}_1\geq 0\\
&\min\limits_{\mathbf{w}_2,b_2} \; \frac{1}{2}||\mathbf{Bw}_2+\mathbf{e}_2b_2||_2^2+c_2\mathbf{e}_1^T\mathbf{q}_2 \nonumber \\
&s.t. \; (\mathbf{Aw}_2+\mathbf{e}_1b_2)+\mathbf{q}_2\geq \mathbf{e}_1, \mathbf{q}_2\geq 0
\end{align}

其中,$\mathbf{A}_{m_1 \times n}=(\mathbf{a}_1^{(1)},\mathbf{a}_2^{(1)},...,\mathbf{a}_{m_1}^{(1)})^T$表示$m_1$个正样本,$\mathbf{B}_{m_2 \times n }=(\mathbf{b}_1^{(2)},\mathbf{a}_2^{(2)},...,\mathbf{a}_{m_2}^{(2)})^T$表示$m_2$个负样本,$\mathbf{e}_1$和$\mathbf{e}_2$表示相应维数的单位变量,$||\cdot||_2$表示L2范数,$\mathbf{q}_1,\mathbf{q}_2$是松弛向量,$c_1,c_2$是非负惩罚系数,分别为正样本和负样本的平衡因子,可以用来解决正负样本个数不同的问题。通过求解以上两个优化问题,可以分别得到两个不平行的超平面:

\begin{align}
\mathbf{x}^T\mathbf{w}_1+b_1=0, \mathbf{x}^T\mathbf{w}_2+b_2=0
\end{align}

当有一个新的点$\mathbf{x}$,计算其到两个超平面的垂直距离,如果它距离超平面$\mathbf{x}^T\mathbf{w}_1+b_1=0$的距离小于它到超平面$\mathbf{x}^T\mathbf{w}_2+b_2=0$的距离,则将该点归入正类,否则它属于负类。

我们也可以得到问题(1)、(2)的Wolfe对偶问题:

\begin{align}
&\max\limits_{\pmb{\alpha}} \; \mathbf{e}_2^T \pmb{\alpha}-\frac{1}{2}\pmb{\alpha}^T\mathbf{G}(\mathbf{H}^T\mathbf{H})^{-1}\mathbf{G}^T\pmb{\alpha} \nonumber \\
&s.t. \; 0 \leq \pmb{\alpha}\leq c_1 \mathbf{e}_2 \\
&\max\limits_{\pmb{\beta}} \; \mathbf{e}_1^T \pmb{\beta}-\frac{1}{2}\pmb{\beta}^T\mathbf{H}(\mathbf{G}^T\mathbf{G})^{-1}\mathbf{H}^T\pmb{\beta} \nonumber \\
&s.t. \; 0 \leq \pmb{\beta} \leq c_2\mathbf{e}_1 
\end{align}

其中$\mathbf{\alpha} \in R^{m_2}$和$\mathbf{\beta}\in R^{m_1}$ 是拉格朗日乘子,可以利用$\mathbf{\alpha} $和$\mathbf{\beta}$得到两个不平行的超平面:

\begin{align}
& \mathbf{z}_1 = (\mathbf{w}^T_1b_1)^T = -(\mathbf{H}^T\mathbf{H})^{-1} \mathbf{G}^
T\pmb{\alpha}\nonumber \\
&\mathbf{z}_2 = (\mathbf{w}^T_2b_2)^T = -(\mathbf{G}^T\mathbf{G})^{-1} \mathbf{H}^T\pmb{\beta}
\end{align}

由于逆矩阵$(\mathbf{H}^T\mathbf{H})^{-1}$和$(\mathbf{G}^T\mathbf{G})^{-1}$可能带来奇异问题,为防止矩阵奇异,可以加入一个正则项$\varepsilon \mathbf{I}$,$\varepsilon$是一个足够小的正数。这样可以保证$(\mathbf{H}^T\mathbf{H}+\varepsilon \mathbf{I})^{-1}$和$(\mathbf{G}^T\mathbf{G}+\varepsilon \mathbf{I})^{-1}$正定,从而不会出现奇异问题。

图1是对TWSVM的几何解释。
\begin{figure}[H]
	\centering 
	\subfigure[]{
    \includegraphics[height=5cm]{TWSVM-img.png}
}
\caption*{图1. TWSVM}
\end{figure}

\end{document}










