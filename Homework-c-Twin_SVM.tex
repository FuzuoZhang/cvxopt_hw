\documentclass{article}
\usepackage[UTF8]{ctex}
\usepackage{amsmath}
\usepackage{amsthm}
\usepackage{amssymb}
\usepackage{graphicx}
\usepackage{color}
%\include{macros}
%\usepackage{floatflt}
%\usepackage{graphics}
%\usepackage{epsfig}
\usepackage[colorlinks,linkcolor=red,anchorcolor=blue,citecolor=green]{hyperref}
\usepackage{epstopdf}

\theoremstyle{definition}
\newtheorem{theorem}{Theorem}[section]
\newtheorem{lemma}[theorem]{Lemma}
\newtheorem{proposition}[theorem]{Proposition}
\newtheorem{corollary}[theorem]{Corollary}

\theoremstyle{definition}
\newtheorem*{definition}{Definition}
\newtheorem*{example}{Example}

\theoremstyle{remark}
\newtheorem*{remark}{Remark}
\newtheorem*{note}{Note}
\newtheorem*{exercise}{Exercise}

\setlength{\oddsidemargin}{-0.25 in}
\setlength{\evensidemargin}{-0.25 in} \setlength{\topmargin}{-0.25
in} \setlength{\textwidth}{7 in} \setlength{\textheight}{8.5 in}
\setlength{\headsep}{0.25 in} \setlength{\parindent}{0 in}
\setlength{\parskip}{0.1 in}

\newcommand{\homework}[4]{
\pagestyle{myheadings} \thispagestyle{plain}
\newpage
\setcounter{page}{1} \setcounter{section}{#4} \noindent
\begin{center}
\framebox{ \vbox{\vspace{2mm} \hbox to 6.28in { {\bf
THU-70250403,~Convex~Optimization~(Fall 2018) \hfill Homework: #4}}
\vspace{6mm} \hbox to 6.28in { {\Large \hfill #1 \hfill} }
\vspace{6mm} \hbox to 6.28in { {\it Lecturer: #2 \hfill} }
\vspace{2mm} \hbox to 6.28in { {\it Student: #3 \hfill} }
\vspace{2mm} } }
\end{center}
\markboth{#1}{#1} \vspace*{4mm} }


\begin{document}

\homework{L1-TWSVM}{Shuning Wang, Li Li \hspace{5mm} {\tt swang@tsinghua.edu.cn \tt
li-li@tsinghua.edu.cn }}{郑洁\hspace{5mm} {\tt j-zheng18@mails.tsinghua.edu.cn} }{2}

\section{L1-TWSVM}
TWSVM有良好的分类性能,已成为数据分类研究的热点。但TWSVM使用对离群值较为敏感的L2范数来度量距离,导致异常观测点可能会对其结果有较大影响。由于L1范数是L2范数距离的鲁棒替代(Ding et al. 2006; Gao 2008; Kwak 2008; Li et al. 2015a; Nie et al. 2015; Wright et al. 2009),本文提出基于L1范数的鲁棒分类器。优化问题如下:
\begin{align}
&\min\limits_{\mathbf{w}_1,b_1} \; \frac{1}{2}||\mathbf{Aw}_1+\mathbf{e}_1b_1||_1+c_1\mathbf{e}_2^T\mathbf{q}_1 \nonumber \\
&s.t.\; -(\mathbf{Bw}_1+\mathbf{e}_2b_1)+\mathbf{q}_1 \geq \mathbf{e}_2,\mathbf{q}_1\geq 0\\
&\min\limits_{\mathbf{w}_2,b_2} \; \frac{1}{2}||\mathbf{Bw}_2+\mathbf{e}_2b_2||_1+c_2\mathbf{e}_1^T\mathbf{q}_2 \nonumber \\
&s.t. \; (\mathbf{Aw}_2+\mathbf{e}_1b_2)+\mathbf{q}_2\geq \mathbf{e}_1, \mathbf{q}_2\geq 0
\end{align}
其中$||\cdot||_1$表示L1范数。在最小化目标函数时,每个平面要尽可能靠近两个分类中的一类,并尽可能远离另一类。由于公式(这里记得改编号!!!!!!!!!)中不等式为非凸约束,具有局部最优解,可以求解得到两个不平行的超平面:
\begin{align}
\mathbf{x}^T\mathbf{w}_1+b_1=0, \mathbf{x}^T\mathbf{w}_2+b_2=0
\end{align}
则原问题可优化为:
\begin{align}
&\min\limits_{\mathbf{w}_1,b_1} \; \frac{1}{2}(\sum_{i=1}^{m_1}\frac{(\mathbf{a}_i^T\mathbf{w}_1+e_1^ib_1)^2}{d_i})+c_1\mathbf{e}_2^T\mathbf{q}_1 \nonumber \\
&s.t.\; -(\mathbf{Bw}_1+\mathbf{e}_2b_1)+\mathbf{q}_1 \geq \mathbf{e}_2,\mathbf{q}_1\geq 0\\
&\min\limits_{\mathbf{w}_2,b_2} \; \frac{1}{2}(\sum_{j=1}^{m_2}\frac{(\mathbf{b}_j^T\mathbf{w}_2+e_2^jb_2)^2}{d_j})+c_2\mathbf{e}_1^T\mathbf{q}_2 \nonumber \\
&s.t. \; (\mathbf{Aw}_2+\mathbf{e}_1b_2)+\mathbf{q}_2\geq \mathbf{e}_1, \mathbf{q}_2\geq 0
\end{align}
其中$d_i=|\mathbf{a}_i^T\mathbf{w}_1+e_1^ib_1|\ne 0$,$d_j=|\mathbf{b}_j^T\mathbf{w}_2+e_2^jb_2|\ne 0$,$e_1^i,e_2^j$分别表示$e_1$的第$i$个元素和$e_2$的第$j$个元素。由于上述两个式子都包含绝对值运算,难以直接求解,本文提出了一种迭代凸优化策略,基本思想为迭代更新增广向量$z_1$直到连续两次迭代式(这里记得改编号!!!!!!!!!)的目标值小于一个固定值(如0.001),则$z_1$为局部最优解。记$z_1^p$为第$p$次迭代结果,则第$p+1$次迭代结果$z_1^{(p+1)}$可等价为下述问题的解:
\begin{align}
&\min\limits_{\mathbf{z}_1} \; \frac{1}{2}(\sum_{i=1}^{m_1}\frac{(\mathbf{h}_i^T\mathbf{z}_1)^2}{d_{1i}})+c_1\mathbf{e}_2^T\mathbf{q}_1 \nonumber \\
&s.t.\; -(\mathbf{Gz}_1+\mathbf{q}_1) \geq \mathbf{e}_2,\mathbf{q}_1\geq 0\\
&\min\limits_{\mathbf{z}_2} \; \frac{1}{2}(\sum_{j=1}^{m_2}\frac{(\mathbf{g}_j^T\mathbf{z}_2)^2}{d_{2j}})+c_2\mathbf{e}_1^T\mathbf{q}_2 \nonumber \\
&s.t. \; (\mathbf{Hz}_2+\mathbf{q}_2)\geq \mathbf{e}_1, \mathbf{q}_2\geq 0
\end{align}
其中$d_{1i}=|\mathbf{h}_i^T\mathbf{z}_1^p|$,$d_{2j}=|\mathbf{g}_j^T\mathbf{z}_2^p|$,$\mathbf{g}_j^T=(\mathbf{b}_j^Te_2^j)$,则公式(19)(20这里记得改编号!!!!!!!!!)可改写为
\begin{align}
&\min\limits_{\mathbf{z}_1} \; \frac{1}{2}\mathbf{z}_1^T\mathbf{H}^T\mathbf{D}_1\mathbf{Hz}_1+c_1\mathbf{e}_2^T\mathbf{q}_1\nonumber\\
&s.t.\; -(\mathbf{Gz}_1+\mathbf{q}_1) \geq \mathbf{e}_2,\mathbf{q}_1\geq 0\\
&\min\limits_{\mathbf{z}_2} \; \frac{1}{2}\mathbf{z}_2^T\mathbf{H}^T\mathbf{D}_2\mathbf{Gz}_2+c_2\mathbf{e}_1^T\mathbf{q}_2\nonumber\\
&s.t. \; (\mathbf{Hz}_2+\mathbf{q}_2)\geq \mathbf{e}_1, \mathbf{q}_2\geq 0
\end{align}
其中$\mathbf{D}_1=diag(1/d_{11},1/d_{12},…,1/d_{1m_1}),\mathbf{D}_2=diag(1/d_{21},1/d_{22},…,1/d_{2m_2})$为对角矩阵。
则问题(21)(22这里记得改编号!!!!!!!!!)等价于:
\begin{align}
&\min\limits_{\mathbf{z}_1} \; \frac{1}{2}||\mathbf{Hz}_1||_1+c_1\mathbf{e}_2^T\mathbf{q}_1\nonumber\\
&s.t.\; -(\mathbf{Gz}_1+\mathbf{q}_1) \geq \mathbf{e}_2,\mathbf{q}_1\geq 0\\
&\min\limits_{\mathbf{z}_2} \; \frac{1}{2}||\mathbf{Gz}_2||_1+c_2\mathbf{e}_1^T\mathbf{q}_2\nonumber\\
&s.t. \; (\mathbf{Hz}_2+\mathbf{q}_2)\geq \mathbf{e}_1, \mathbf{q}_2\geq 0
\end{align}
公式(14这里记得改编号!!!!!!!!!)是不等式约束(非凸)的凸优化问题,因此它存在解析解。其拉格朗日函数为:
\begin{align}
\pmb{L}_1(\mathbf{w}_1,b_1,\mathbf{q}_1,\pmb{\alpha},\pmb{\beta}) &=\frac{1}{2}(\mathbf{Aw}_1+\mathbf{e}_1b_1)^T\mathbf{D}_1(\mathbf{Aw}_1+\mathbf{e}_1b_1)\nonumber\\
&+c_1\mathbf{e}_2^T\mathbf{q}_1-\pmb{\alpha}^T(-(\mathbf{Bw}_1+\mathbf{e}_2b_1)+\mathbf{q}_1-\mathbf{e}_2)-\pmb{\beta}^T\mathbf{q}_1
\end{align}
其中$\pmb{\alpha}=(\alpha_1,\alpha_2,\alpha_3,…,\alpha_{m_2})^T, \pmb{\beta}=(\beta_1,\beta_2,\beta_3,…,\beta_{m_1})^T$为拉格朗日乘子,$\pmb{\alpha}\geq 0,\pmb{\beta}\geq 0$,令$\pmb{L}_1$对$\mathbf{w}_1,b_1,\mathbf{q}_1$的偏导分别为0,可得Karush-Kuhn-Tucker (KKT)条件为:
\begin{align}
&\frac{\partial{L}}{\partial{\mathbf{w}_1}}=\mathbf{A}^T\mathbf{D}_1(\mathbf{Aw}_1+\mathbf{e}_1b_1)+\mathbf{B}^T\pmb{\alpha}=0\\
&\frac{\partial{L}}{\partial{b_1}}=\mathbf{e}_1^T\mathbf{D}_1(\mathbf{Aw}_1+\mathbf{e}_1b_1)+\mathbf{e}_2\pmb{\alpha}=0\\
&\frac{\partial{L}}{\partial{\mathbf{q}_1}}=c_1\mathbf{e}_2-\pmb{\alpha}-\pmb{\beta}=0\\
&-(\mathbf{Bw}_1+\mathbf{e}_2b_1)+\mathbf{q}_1 \geq \mathbf{e}_2, \mathbf{q}_1\geq 0\\
&\pmb{\alpha}^T(-(\mathbf{Bw}_1+\mathbf{e}_2b_1)+\mathbf{q}_1-\mathbf{e}_2)=0,\pmb{\beta}^T\mathbf{q}_1=0
\end{align}
从式(28这里记得改编号!!!!!!!!!)可推出$0\le \pmb{\alpha} \le c_1\mathbf{e}_2$,结合公式(26)(27)可得:
\begin{align}
\mathbf{A}^T\mathbf{e}_1^T\mathbf{D}_1(\mathbf{Ae}_1)(\mathbf{w}_1b_1)^T+(\mathbf{B}^T\mathbf{e}_2^T)\pmb{\alpha}=0
\end{align}
结合之前定义的矩阵$(\mathbf{H,G})$及增广向量$(\mathbf{z}_1,\mathbf{z}_2)$,可以得到
\begin{align}
\mathbf{H}^T\mathbf{D}_1^p\mathbf{Hz}_1^{(p+1)}+\mathbf{G}^T\pmb{\alpha}=0
\end{align}
即
\begin{align}
\mathbf{z}_1^{(p+1)}=-(\mathbf{H}^T\mathbf{D}_1^p\mathbf{H})^{-1}\mathbf{G}^T\pmb{\alpha}
\end{align}
由于$(\mathbf{H}^T\mathbf{D}_1^p\mathbf{H})^{-1}$为半正定矩阵,因此可能得到不稳定或不准确的解,在实际应用中,本文使用正则化方法(Jayadeva and Chandra (2007), Mangasarian andWild (2006))解决这个问题。$(\mathbf{H}^T\mathbf{D}_1^p\mathbf{H}+\varepsilon \mathbf{I})$为正定矩阵(其中$\varepsilon$为一个小扰动),不受奇点的影响。则逆矩阵$(\mathbf{H}^T\mathbf{D}_1^p\mathbf{H})^{-1}$可由$(\mathbf{H}^T\mathbf{D}_1^p\mathbf{H}+\varepsilon \mathbf{I})$代替,因此$\mathbf{z}_1^{(p+1)}$可推导为:
\begin{align}
\mathbf{z}_1^{(p+1)}=-(\mathbf{H}^T\mathbf{D}_1^p\mathbf{H}+\varepsilon \mathbf{I})^{-1}\mathbf{G}^T\pmb{\alpha}
\end{align}
同样的
\begin{align}
\mathbf{z}_2^{(p+1)}=-(\mathbf{G}^T\mathbf{D}_2^p\mathbf{G}+\varepsilon \mathbf{I})^{-1}\mathbf{H}^T\pmb{\beta}
\end{align}
将增广向量$\mathbf{z}_1^{(p+1)},\mathbf{z}_2^{(p+1)}$分别代入拉格朗日函数中。在KKT条件下,原问题(14)(15这里记得改编号!!!!!!!!!)转变为Wolfe对偶问题:
\begin{align}
&\max \limits_{\pmb{\alpha}} \; \mathbf{e}_2^T\pmb{\alpha}-\frac{1}{2}\pmb{\alpha}^T\mathbf{G}(\mathbf{H}^T\mathbf{D}_1\mathbf{H})^{-1}\mathbf{G}^T\pmb{\alpha}\nonumber\\
&s.t. 0\le \pmb{\alpha} \le c_1\mathbf{e}_2\\
&\max \limits_{\pmb{\beta}} \; \mathbf{e}_1^T\pmb{\beta}-\frac{1}{2}\pmb{\beta}^T\mathbf{H}(\mathbf{G}^T\mathbf{D}_2\mathbf{G})^{-1}\mathbf{H}^T\pmb{\beta}\nonumber\\
&s.t. 0\le \pmb{\beta} \le c_2\mathbf{e}_1
\end{align}
通过求解对偶问题可得拉格朗日乘子$\pmb{\alpha} \in \pmb{R}^{m_2\times 1},\pmb{\beta} \in \pmb{R}^{m_1\times 1}$以及权向量$\mathbf{w}_1,\mathbf{w}_2$、偏差$b_1,b_2$,即获得两个不平行的超平面。
对于新加入的点$\mathbf{x}\in \mathbf{R}^n$,根据决策方程$f(\mathbf{x})$(选择最近的超平面)将其分配到对应类别中
\begin{align}
f(\mathbf{x})=arg \min\limits_{i=1,2} \;(|\mathbf{x}^T\mathbf{w}_i+b_i|/|||\mathbf{w}_i|)
\end{align}
其中$|\cdot|$表示取绝对值。
式(14这里记得改编号!!!!!!!!!)中的目标函数为非凸约束的凸问题,因此$\mathbf{z}_1^{(p+1)}$为此问题的局部最优解。而在式(33这里记得改编号!!!!!!!!!)中,$\mathbf{D}_1^p$依赖于$\mathbf{z}_1^{(p+1)}$,因此它是一个未知变量,可看作(14)中目标的隐变量,可以同相同的迭代算法交替优化求解。我们根据前一次迭代结果$\mathbf{z}_1^{(p+1)}$来更新$\mathbf{D}_1^p$,又通过$\mathbf{D}_1^p$来改变$\mathbf{z}_1^{(p+1)}$,增加p直到连续两次迭代结果小于一个固定值。此外,适当的初始化可有效加快算法收敛速度。本文通过求解公式(8)(9这里记得改编号!!!!!!!!)得到初始解,仿真结果较优。算法1总结了L1-TWSVM的迭代过程。
\end{document}